\section{Appendix}
\subsection{Running time}
\begin{theorem}
	Brute Force Graph Identification has runtime $\mathcal{O}((n+2)!)$.
\end{theorem}
\begin{proof}
	We write out the steps and the frequency of each step. Observe,
	\[\begin{tabular}{|c|c|c|c|}
		\hline
		& Steps&Frequency&Total\\
		\hline
		Initialize&$(n+1)!+n^2$&1&$n^2+(n+1)!$\\
		\hline
		Iterate&${n\choose 2}$&$n!$&${n\choose2}(n!)$\\
		\hline
		Output&1&1&1\\
		\hline
	\end{tabular}\]
	Summing the entries of the Total column, we have\[f(n)=n^2+(n+1)!+{n\choose2}(n!)+1.\]
	Let $n_0=10$ and suppose $n>n_0$. Observe,
	\begin{align*}
		f(n)&=(n+1)!+{n\choose2}(n!)+1=(n+1)!+\frac{n!n!}{(n-2)!2}+1=(n+1)!+\frac{n!(n)(n-1)}{2}+1\\
		&\le(n+1)!+n!(n)(n-1)\le(n+1)!+(n+2)!\le2((n+1)!).
	\end{align*}
	Hence, $f(n)=\mathcal{O}((n+2)!)$ where $c=2$.
\end{proof}
\subsection{Babai's Algorithm}
The following is an overview of \cite{babai2016,babai2018}. Babai's goal is to use a "Divide-and-Conquer" recursive method, pioneered in \cite{luks1982}. 
\begin{definition}
	Let $\mathcal{A}$ be an algorithm with an input of size $n$ and $q(n)$ be the number of smaller instances $n$ is reduced to. Ignoring the additive cost of assembling all information from smaller instances,
	\[f(\mathcal{A},n)\le q(n)f(\mathcal{A},0.9n).\] 
\end{definition}
Here is a high level view taken from \cite{babai2018,babai2016} on how his algorithm divides the GI into smaller instances (see Appendix for most definitions):
\begin{enumerate}
	\item Reduce GI to the more general String Isomorphism Problem (SI) (see Section III)
	\item Solve SI in quasipolynomial time via:
	\subitem{a.} Follow \cite{luks1982} algorithm until hitting a bottlenock
	\subitem{b.} At a bottleneck, an "ideal domain" $\Gamma$ is revealed. 
	\subsubitem{i.}Use a group theoretic algorithm called "Local Certificates" to construct a canonical structure on the domain.
	\subsubitem{ii.} Use combinatroial partitioning algorithms to reducee the size of $\Gamma$.
	\subsubitem{iii.} Use individualization (i.e., Weisfeiler-Lehman) to individualize each element of $\Gamma$, which reduces the size of the input string.
\end{enumerate}