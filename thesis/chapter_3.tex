\section{The String Isomorphism Problem}
\subsection{A graph as a string}
Following Luks [1982], a graph can be represented as a binary string using the indicator function for adjacency relations in the graph. Let $\Omega=[1,\dots,n]$. Let $G=\{\Omega,E\}$ be an undirected, simple graph. Let ${\Omega\choose 2}$ denote the set of all unordered pairs in $\Omega$ (Babai 2018). Let $\delta_G:{\Omega\choose 2}\rightarrow\{0,1\}$ be the indicator function for the adjacency relations in $G$, defined by
\[\delta_G(\{x,y\})=\begin{cases}
1,&\{x,y\}\in E(G)\\
0,&\{x,y\}\notin E(G)
\end{cases}.
\]
Then $\delta_G$ is the binary string representation of $G$.
\begin{definition}\label{def:def444}
Sym$(\Omega) $ (denoted $S_\Omega)$ is the group of all bijections $f:\Omega\rightarrow\Omega$.
\end{definition}
From Definition~\ref{def:def333}, we know that two graphs $G$ and $G'$ are isomorphic if there exists a bijection $f:V(G)\rightarrow V(G')$ such that for all $u,v\in V(G)$, $\{u,v\}\in E(G)$ if and only if $\{f(u),f(v)\}\in E(G')$. Since all of the graphs we are discussing are labelled using $[n]$, every isomorphism between two graphs is a function from $S_n$. In order to define the string isomorphism problem, the following definition is important.
\begin{definition}$\;$\\
$S_n^{(2)}=\left\{\sigma\in\text{Sym }\left({\Omega\choose2}\right):\exists f\in S_n\text{ s.t. }\forall \omega=\{u,v\}\in{\Omega\choose2},\;\sigma(\omega)=\{f(u),f(v)\} \right\}.$
\end{definition}
Notice that $S_n^{(2)}\subseteq$ Sym$({\Omega\choose 2})$. There is an added restriction on the bijections in $S_n^{(2)}$ compared to those in Sym$({\Omega\choose 2})$. Every permutation of unordered pairs (i.e. permutation of edges) in $S_n^{(2)}$ can be obtained using a permutation in $S_n$ (i.e. a permutation of the vertices). When $0<n\le 3$, $S_n^{(2)}=$Sym$({\Omega\choose 2})$. When $n>3$, the two groups are no longer equal. Consider the following example.
\begin{example}\label{ex:ex223}
Let \[\sigma=\begin{pmatrix}\{1,2\}&\{1,3\}&\{1,4\}&\{2,3\}&\{2,4\}&\{3,4\}\\
\{1,3\}&\{1,2\}&\{1,4\}&\{2,3\}&\{2,4\}&\{3,4\}
\end{pmatrix}.\] 
Notice that $\sigma\in$ Sym$\left({4\choose 2}\right)$. To see that $\sigma\notin S_4^{(2)}$, we must show that there is no function $f\in S_4$ such that for all $u,v\in [4]$, $\sigma(\{u,v\})=\{f(u),f(v)\}$. Although there are 24 elements in $S_4$, there is only one element in $S_4$ (see $f_1$ below) that results in mapping $\{1,2\}\rightarrow \{1,3\}$ and $\{1,3\}\rightarrow\{1,2\}$. We have

\[f_1=\begin{pmatrix}1&2&3&4\\
1&3&2&4
\end{pmatrix}.\]
Notice that $\sigma (\{3,4\})=\{3,4\}\ne \{2,4\}=\{f_1(3),f_1(4)\}$. Since this is the only possible bijection for $\sigma$ to satisfy the conditions of $S_4^{(2)}$, we know $\sigma\notin S_4^{(2)}$. To make this point more clear, consider the graphs on four vertices below. 
\begin{figure}[H]
\begin{subfigure}{.5\textwidth}
% https://q.uiver.app/?q=WzAsNCxbMCwwLCIxXFxidWxsZXQiXSxbMSwwLCJcXGJ1bGxldCAyIl0sWzEsMSwiXFxidWxsZXQgMyJdLFswLDEsIjRcXGJ1bGxldCJdLFswLDEsIiIsMCx7InN0eWxlIjp7ImhlYWQiOnsibmFtZSI6Im5vbmUifX19XSxbMSwyLCIiLDAseyJzdHlsZSI6eyJoZWFkIjp7Im5hbWUiOiJub25lIn19fV0sWzIsMCwiIiwwLHsic3R5bGUiOnsiaGVhZCI6eyJuYW1lIjoibm9uZSJ9fX1dLFsyLDMsIiIsMCx7InN0eWxlIjp7ImhlYWQiOnsibmFtZSI6Im5vbmUifX19XV0=
\[\begin{tikzcd}
	1\bullet & {\bullet 2} \\
	4\bullet & {\bullet 3}
	\arrow[no head, from=1-1, to=1-2]
	\arrow[no head, from=1-2, to=2-2]
	\arrow[no head, from=2-2, to=1-1]
	\arrow[no head, from=2-2, to=2-1]
\end{tikzcd}\]
\caption{$G$}
\end{subfigure}
\begin{subfigure}{.5\textwidth}
% https://q.uiver.app/?q=WzAsNCxbMCwwLCIxXFxidWxsZXQiXSxbMSwwLCJcXGJ1bGxldCAzIl0sWzEsMSwiXFxidWxsZXQgMiJdLFswLDEsIjRcXGJ1bGxldCJdLFswLDEsIiIsMCx7InN0eWxlIjp7ImhlYWQiOnsibmFtZSI6Im5vbmUifX19XSxbMSwyLCIiLDAseyJzdHlsZSI6eyJoZWFkIjp7Im5hbWUiOiJub25lIn19fV0sWzIsMCwiIiwwLHsic3R5bGUiOnsiaGVhZCI6eyJuYW1lIjoibm9uZSJ9fX1dLFsyLDMsIiIsMCx7InN0eWxlIjp7ImhlYWQiOnsibmFtZSI6Im5vbmUifX19XV0=
\[\begin{tikzcd}
	1\bullet & {\bullet 3} \\
	4\bullet & {\bullet 2}
	\arrow[no head, from=1-1, to=1-2]
	\arrow[no head, from=1-2, to=2-2]
	\arrow[no head, from=2-2, to=1-1]
	\arrow[no head, from=2-2, to=2-1]
\end{tikzcd}\]
\caption{$f_1(V(G))$}
\end{subfigure}
\caption{$G$ and the result of applying $f_1$ to V(G).}
\end{figure}
Notice that $\{2,4\}\in E(f_1(V(G)))$ but $\{2,4\}\notin E(G)$ 
\end{example}




\begin{definition}\label{def:def225}
Let $\delta_1$ and $\delta_2$ be string representations of undirected, simple graphs $G_1$ and $G_2$, respectively. Then $\delta_1$ and $\delta_2$ are $S_n^{(2)}$\textit{-isomorphic}, denoted $\delta_1\cong \delta_2$, if there exists $\sigma\in S_n^{(2)}$ such that $\delta_1\circ\sigma=\delta_2$.
\end{definition}
Given binary strings $\delta_1$ and $\delta_2$, both with length $n$, the \textit{string isomorphism problem} asks: Does there exist $\sigma\in S_n^{(2)}$ such that $\delta_1\circ\sigma=\delta_2$? The following example answers this question for a binary string with length 6.
\begin{example}\label{ex:ex221}
\begin{figure}[H]
\begin{subfigure}{.5\textwidth}
% https://q.uiver.app/?q=WzAsNCxbMCwwLCIxXFxidWxsZXQiXSxbMSwwLCJcXGJ1bGxldCAyIl0sWzEsMSwiXFxidWxsZXQgMyJdLFswLDEsIjRcXGJ1bGxldCJdLFswLDEsIiIsMCx7InN0eWxlIjp7ImhlYWQiOnsibmFtZSI6Im5vbmUifX19XSxbMSwzLCIiLDAseyJzdHlsZSI6eyJoZWFkIjp7Im5hbWUiOiJub25lIn19fV1d
\[\begin{tikzcd}
	1\bullet & {\bullet 2} \\
	4\bullet & {\bullet 3}
	\arrow[no head, from=1-1, to=1-2]
	\arrow[no head, from=1-2, to=2-1]
\end{tikzcd}\]
\caption{$G_1$}
\end{subfigure}
\begin{subfigure}{.5\textwidth}
% https://q.uiver.app/?q=WzAsNCxbMCwwLCIxXFxidWxsZXQiXSxbMSwwLCJcXGJ1bGxldCAyIl0sWzEsMSwiXFxidWxsZXQgMyJdLFswLDEsIjRcXGJ1bGxldCJdLFswLDEsIiIsMCx7InN0eWxlIjp7ImhlYWQiOnsibmFtZSI6Im5vbmUifX19XSxbMCwyLCIiLDIseyJzdHlsZSI6eyJoZWFkIjp7Im5hbWUiOiJub25lIn19fV1d
\[\begin{tikzcd}
	1\bullet & {\bullet 2} \\
	4\bullet & {\bullet 3}
	\arrow[no head, from=1-1, to=1-2]
	\arrow[no head, from=1-1, to=2-2]
\end{tikzcd}\]
\caption{$G_2$}
\end{subfigure}
\caption{Two isomorphic graphs on 4 vertices.}
\end{figure}
Consider graphs $G_1$ and $G_2$ in Figure 3qgis
.2. In this case, we consider $\Omega=[1,2,3,4]$. A bijection that produces an isomorphism between $G_1$ and $G_2$ is $f_2$,
\[f_2=\begin{pmatrix}1&2&3&4\\
2&1&4&3
\end{pmatrix}.\]
If we apply $f_2$ to $V(G_1)$, the resulting graph is $G_3$ (Figure 2.3).
\begin{figure}
% https://q.uiver.app/?q=WzAsNCxbMCwwLCIyXFxidWxsZXQiXSxbMSwwLCJcXGJ1bGxldCAxIl0sWzEsMSwiXFxidWxsZXQgNCJdLFswLDEsIjNcXGJ1bGxldCJdLFswLDEsIiIsMCx7InN0eWxlIjp7ImhlYWQiOnsibmFtZSI6Im5vbmUifX19XSxbMSwzLCIiLDAseyJzdHlsZSI6eyJoZWFkIjp7Im5hbWUiOiJub25lIn19fV1d
\[\begin{tikzcd}
	2\bullet & {\bullet 1} \\
	3\bullet & {\bullet 4}
	\arrow[no head, from=1-1, to=1-2]
	\arrow[no head, from=1-2, to=2-1]
\end{tikzcd}\]
\caption{$G_3$, with vertex set $f_2(V(G_1))$.}
\end{figure}
We now construct the binary string representation of each graph. We have \[\delta_{G_1}=\begin{pmatrix}\{1,2\}&\{1,3\}&\{1,4\}&\{2,3\}&\{2,4\}&\{3,4\}\\
1&0&0&0&1&0
\end{pmatrix}\] and 
\[\delta_{G_2}=\begin{pmatrix}\{1,2\}&\{1,3\}&\{1,4\}&\{2,3\}&\{2,4\}&\{3,4\}\\
1&1&0&0&0&0
\end{pmatrix}.\] We also have
\[\text{Sym }(\Omega)=S_4.\]
To show that $\delta_{G_1}\cong\delta_{G_2}$, we must find $\sigma\in S_4^{(2)}$ such that $\delta_{G_1}\circ \sigma =\delta_{G_2}$. We simply record the effect on $E(G_1)$ of applying $f_2$ to $V(G_1)$. We have
\[\sigma =\begin{pmatrix}\{1,2\}&\{1,3\}&\{1,4\}&\{2,3\}&\{2,4\}&\{3,4\}\\
\{1,2\}&\{2,4\}&\{2,3\}&\{1,4\}&\{1,3\}&\{3,4\}
\end{pmatrix}.\]
Then 
\[\delta_{G_1}\circ \sigma=\begin{pmatrix}\{1,2\}&\{1,3\}&\{1,4\}&\{2,3\}&\{2,4\}&\{3,4\}\\
\{1,2\}&\{2,4\}&\{2,3\}&\{1,4\}&\{1,3\}&\{3,4\}\\
1&1&0&0&0&0
\end{pmatrix}=\delta_{G_2},\]
as desired. Hence, $\delta_{G_1}\cong\delta_{G_2}$.
\end{example}

\begin{lemma}
$S_n^{(2)}$ is a subgroup of Sym$({[n]\choose 2})$. 
\end{lemma}
\begin{proof}
Let $\sigma,\eta\in S_n^{(2)}$. Since $S_n^{(2)}\subseteq$ Sym$\left({[n]\choose 2}\right)$, $\sigma,\eta\in$ Sym$\left({[n]\choose 2}\right)$. Moreover, Sym$\left({[n]\choose 2}\right)$ is a group. Hence, $\sigma\circ\eta\in$ Sym$\left({[n]\choose 2}\right)$. Now, by definition there exists $f_\sigma,\;f_\eta\in S_n$ such that for all $\omega=\{u,v\}\in$ Sym$({[n]\choose 2})$, $\sigma(\{u,v\})=\{f_\sigma(u),f_\sigma(v)\}$ and $\eta(\{u,v\})=\{f_\eta(u),f_\eta(v)\}.$ Since $f_\sigma,f_\eta\in S_n$, $f_\sigma\circ f_\eta\in S_n$. Let $\omega=\{u,v\}\in$ Sym$({[n]\choose 2})$. Consider $\sigma\circ\eta(\{u,v\})$. We have
\[\sigma\circ\eta(\{u,v\})=\sigma(\{f_\eta(u),f_\eta(v)\})=\{f_\sigma\circ f_\eta(u),f_\sigma\circ f_\eta(v)\}.\] 
Hence, $\sigma\circ\eta\in S_n^{(2)}$. Lastly, consider $f_\sigma^{-1}\in S_n$. Let $\sigma^{-1}(\{u,v\})=\{f_\sigma^{-1}(u),f_\sigma^{-1}(v)\}$. Hence, $\sigma^{-1}\in S_n^{(2)}$. Observe,
\[\sigma\circ\sigma^{-1}(\{u,v\})=\sigma(\{f_\sigma^{-1}(u),f_\sigma^{-1}(v)\})=\{f_\sigma\circ f_\sigma^{-1}(u),f_\sigma\circ f_\sigma^{-1}(v)\}=\{u,v\}.\] By a similar argument, $\sigma^{-1}\sigma(\{u,v\})=\{u,v\}$. Hence, $S_n^{(2)}$ is a subgroup of Sym$({[n]\choose 2})$.
\end{proof}
\begin{lemma}
Two graphs are isomorphic if and only if their string representations are $S_n^{(2)}$-isomorphic.
\end{lemma}
\begin{proof}
Let $\delta_{G_1}$ and $\delta_{G_2}$ be binary string representations of undirected graphs $G_1$ and $G_2$, respectively. \\
$\Rightarrow$\\
Suppose $G_1\cong G_2$. By Definition~\ref{def:def333}, there exists bijection $g:V(G_1)\rightarrow V(G_2)$ such that for all $u,v\in V(G_1)$, $\{u,v\}\in E(G_1)$ if and only if $\{g(u),g(v)\}\in E(G_2)$. Notice that $g\in S_n$. Let $\sigma:{[n]\choose 2}\rightarrow {[n]\choose 2}$ where $\sigma(\{u,v\})=\{g(u),g(v)\}$. Then $\sigma\in S_n^{(2)}$. Let $u,v\in [n]$. Consider $\delta_{G_2}\circ\sigma(\{u,v\})$. We have
\[\delta_{G_2}\circ\sigma(\{u,v\})=\delta_{G_2}(\{g(u),g(v)\})=\delta_{G_1}(\{u,v\}).\] The last equality follows because $\{u,v\}\in E(G_1)$ if and only if $\{g(u),g(v)\}\in E(G_2)$. This means  $\delta_{G_1}(\{u,v\})=1$ if and only if $\delta_{G_2}(\{g(u),g(v)\})=1$. Hence, $\delta_{G_2}\cong \delta_{G_1}$.\\
$\Leftarrow$\\
Suppose $\delta_{G_1}\cong \delta_{G_2}$. Note that for any $\sigma\in S_n^{(2)}$, there exist $f_\sigma\in S_n$ such that $\sigma(\{u,v\})=\{f_\sigma(u),f_\sigma(v)\}$ for any $\{u,v\}\in{[n]\choose2}$. Since $\delta_{G_1}\cong\delta_{G_2}$, by Definition~\ref{def:def225}, there exists $\hat\sigma\in S_n^{(2)}$ such that $\delta_{G_1}\circ \hat\sigma=\delta_{G_2}$. We have
\[\delta_{G_2}(\{u,v\})=\delta_{G_1}\circ \hat\sigma(\{u,v\})=\delta_{G_1}(\{f_{\hat\sigma}(u),f_{\hat\sigma}(v)\}).\]
This means that $\delta_{G_2}(\{u,v\})=1$ if and only if $\delta_{G_1}(\{f_{\hat\sigma(u)},f_{\hat\sigma(v)}\})=1$. In other words, $\{u,v\}\in E(G_2)$ if and only if $\{f_{\hat\sigma(u)},f_{\hat\sigma(v)}\}\in E(G_1)$. Hence, $G_1\cong G_2$. 
\end{proof}


